\section*{Diskuze}
Naměřená závislost prvního vzorku odpovídá teoretické pro štítkovou hodnotu velmi přesně pro obě polarizace.

U druhého vzorku pro p-polarizaci odpovídá také velice přesně. Pro s-polarizaci je už ale znatelná odchylka, sic ne přílíš veliká. Navíc naměřená závislost je v některých pásmech vyšší a v některých nižší než teoretická (viz obrázek \ref{g:2}), takže není možné dojít k uspokojivému výsledku fitováním indexu lomu, protože funkce $R_s$ v \eqref{e:fresnely} je v $N$ všude rostoucí. Metoda nejmenších čtverců dává $N = \num{1.765}$.


Určení Brewsterova úhlu bylo překvapivě přesné. Dalo by se zpřesnit zvýšením citlivosti detektoru/zesílení signálu v okolí Brewsterova úhlu. Odečítali jsme totiž pouze jednu platnou číslici. Vztah \eqref{e:brewster} je velmi citlivý na chybu Brewsterova úhlu. Přesto se změřený index lomu velmi dobře shodoval s tím uvedeným na štítku.