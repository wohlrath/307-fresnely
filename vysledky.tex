\section*{Výsledky měření}
Intenzitu uvádíme v relativních jednotkách odpovídajících napětí na voltmetru připojenému na detektor.

Všechny uvedené odchylky jsou standardní ($P\approx\SI{68}{\percent}$).

Našli jsme směr snadného průchodu polarizátoru. Pod Brewsterovým úhlem jsme hledali minimum intenzity otáčením polarizátoru. Hledání jsme provedeli třikrát, výsledné úhly natočení polarizátoru byly \SI{252}{\degree}, \SI{252}{\degree} a \SI{256}{\degree}.

Závislost intenzity odraženého světla na úhlu dopadu jsme změřili se dvěma vzorky, jejichž indexy lomu byly podle štítku \num{1.509} pro \emph{první vzorek} a \num{1.8051} pro \emph{druhý vzorek}. Naměřené hodnoty jsou uvedené v přiložené tabulce a zaneseny do grafů \ref{g:1} a \ref{g:2} spolu s teoretickou závislostí pro hodnoty na štítku. 

Změřili jsme Brewsterův úhel pro oba vzorky
\begin{equation*}
\theta_{B1} = \SI{56.5(5)}{\degree} \qquad \qquad \qquad \theta_{B2} = \SI{61(1)}{\degree} \,.
\end{equation*}
Standardní odchylku jsme odhadli vzhledem k nízkému kontrastu v okolí Brewsterova úhlu.
Indexy lomu obou vzorků vypočtené z Brewsterova úhlu pomocí \eqref{e:brewster} jsou
\begin{equation*}
n_1 = \num{1.51(3)} \qquad \qquad \qquad n_2 = \num{1.80(7)} \,.
\end{equation*}
Standardní relativní chyba indexu lomu prvního resp. druhého vzorku je tedy \SI{2}{\percent} resp. \SI{4}{\percent}.

\begin{graph}[p] 
\centering
\input{1.tex}
\caption{První vzorek s indexem lomu \num{1.509}}
\label{g:1}
\end{graph}


\begin{graph}[p] 
\centering
\input{2.tex}
\caption{Druhý vzorek s indexem lomu \num{1.8051}}
\label{g:2}
\end{graph}