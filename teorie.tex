\section*{Teoretická část}
Fresnelovy vzorce dávájí odrazivost pro s-polarizaci (příčná) a p-polarizaci (podélná) \cite{maly}

\begin{equation} \label{e:fresnely}
R_s=\left| \frac{\cos\theta_i-N\cos\theta_t}{\cos\theta_i+N\cos\theta_t}\right|^2 \qquad \qquad R_p=\left| \frac{\cos\theta_t-N\cos\theta_i}{\cos\theta_t+N\cos\theta_i}\right|^2 \,,
\end{equation}
kde $N$ je relativní index lomu, $\theta_i$ je dopadový úhel paprsku a
\begin{equation*}
\theta_t=\sqrt{1-\frac{\sin^2\theta_i}{N^2}} \,.
\end{equation*}

Odrazivost pro p-polarizaci vymizí, pokud se úhel dopadu rovná tzv. Brewsterovu úhlu, pro který platí
\begin{equation} \label{e:brewster}
\tan\theta_B = N \,.
\end{equation}

Úhel $\gamma$ mezi paprskem a rovinou vzorku, který jsme odečítali na stupnici goniometru, je s úhlem dopadu ve vztahu
\begin{equation*}
\gamma + \theta_i = \SI{90}{\degree}
\end{equation*}

