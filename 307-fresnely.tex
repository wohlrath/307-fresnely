\documentclass[a4paper]{article}

\usepackage[czech]{babel} %https://github.com/michal-h21/biblatex-iso690
\usepackage[
   backend=biber      % if we want unicode 
  ,style=iso-numeric % or iso-numeric for numeric citation method          
  ,babel=other        % to support multiple languages in bibliography
  ,sortlocale=cs_CZ   % locale of main language, it is for sorting
  ,bibencoding=UTF8   % this is necessary only if bibliography file is in different encoding than main document
]{biblatex}

\usepackage[utf8]{inputenc}
\usepackage{fancyhdr}
\usepackage{amsmath}
\usepackage{amssymb}
\usepackage[left=2cm,right=2cm,top=2.5cm,bottom=2.5cm]{geometry}
\usepackage{graphicx}
\usepackage{pdfpages}
\usepackage{url}

\usepackage{siunitx}
\sisetup{locale = DE, separate-uncertainty = true    } %kdybych chtel +/-

\usepackage{float}
\newfloat{graph}{htbp}{grp}
\floatname{graph}{Graf}
\newfloat{tabulka}{htbp}{tbl}
\floatname{tabulka}{Tabulka}

\renewcommand{\thefootnote}{\roman{footnote}}

\pagestyle{fancy}
\lhead{Praktikum III - (7) Ověření Fresnelových vzorců}
\rhead{Vladislav Wohlrath}
\author{Vladislav Wohlrath}

\bibliography{source}


\begin{document}

\begin{titlepage}
\includepdf[pages={1}]{./graficos/tit.pdf}
\end{titlepage}

\section*{Pracovní úkoly}
\begin{enumerate}
\item Najděte směr snadného průchodu polarizátoru užívaného v aparatuře.
\item Ověřte, že zdroj světla je polarizován kolmo k vodorovné rovině.
\item Na přiložených vzorcích proměřte závislost intenzity odraženého světla na úhlu dopadu pro \emph{TE} i \emph{TM} polarizaci.
\item Naměřené výsledky porovnejte s teoretickým průběhem závislostí.
\item Určete indexy lomů měřených vzorků a jejich relativní chybu.

\end{enumerate}

%Teoretická část
\section*{Teoretická část}
Fresnelovy vzorce dávájí odrazivost pro s-polarizaci (příčná) a p-polarizaci (podélná) \cite{maly}

\begin{equation} \label{e:fresnely}
R_s=\left| \frac{\cos\theta_i-N\cos\theta_t}{\cos\theta_i+N\cos\theta_t}\right|^2 \qquad \qquad R_p=\left| \frac{\cos\theta_t-N\cos\theta_i}{\cos\theta_t+N\cos\theta_i}\right|^2 \,,
\end{equation}
kde $N$ je relativní index lomu, $\theta_i$ je dopadový úhel paprsku a
\begin{equation*}
\theta_t=\sqrt{1-\frac{\sin^2\theta_i}{N^2}} \,.
\end{equation*}

Odrazivost pro p-polarizaci vymizí, pokud se úhel dopadu rovná tzv. Brewsterovu úhlu, pro který platí
\begin{equation} \label{e:brewster}
\tan\theta_B = N \,.
\end{equation}

Úhel $\gamma$ mezi paprskem a rovinou vzorku, který jsme odečítali na stupnici goniometru, je s úhlem dopadu ve vztahu
\begin{equation*}
\gamma + \theta_i = \SI{90}{\degree}
\end{equation*}



%Výsledky měření
\section*{Výsledky měření}
Intenzitu uvádíme v relativních jednotkách odpovídajících napětí na voltmetru připojenému na detektor.

Všechny uvedené odchylky jsou standardní ($P\approx\SI{68}{\percent}$).

Našli jsme směr snadného průchodu polarizátoru. Pod Brewsterovým úhlem jsme hledali minimum intenzity otáčením polarizátoru. Hledání jsme provedeli třikrát, výsledné úhly natočení polarizátoru byly \SI{252}{\degree}, \SI{252}{\degree} a \SI{256}{\degree}.

Závislost intenzity odraženého světla na úhlu dopadu jsme změřili se dvěma vzorky, jejichž indexy lomu byly podle štítku \num{1.509} pro \emph{první vzorek} a \num{1.8051} pro \emph{druhý vzorek}. Naměřené hodnoty jsou uvedené v přiložené tabulce a zaneseny do grafů \ref{g:1} a \ref{g:2} spolu s teoretickou závislostí pro hodnoty na štítku. 

Změřili jsme Brewsterův úhel pro oba vzorky
\begin{equation*}
\theta_{B1} = \SI{56.5(5)}{\degree} \qquad \qquad \qquad \theta_{B2} = \SI{61(1)}{\degree} \,.
\end{equation*}
Standardní odchylku jsme odhadli vzhledem k nízkému kontrastu v okolí Brewsterova úhlu.
Indexy lomu obou vzorků vypočtené z Brewsterova úhlu pomocí \eqref{e:brewster} jsou
\begin{equation*}
n_1 = \num{1.51(3)} \qquad \qquad \qquad n_2 = \num{1.80(7)} \,.
\end{equation*}
Standardní relativní chyba indexu lomu prvního resp. druhého vzorku je tedy \SI{2}{\percent} resp. \SI{4}{\percent}.

\begin{graph}[p] 
\centering
\input{1.tex}
\caption{První vzorek s indexem lomu \num{1.509}}
\label{g:1}
\end{graph}


\begin{graph}[p] 
\centering
\input{2.tex}
\caption{Druhý vzorek s indexem lomu \num{1.8051}}
\label{g:2}
\end{graph}

%Diskuze výsledků
\section*{Diskuze}
Naměřená závislost prvního vzorku odpovídá teoretické pro štítkovou hodnotu velmi přesně pro obě polarizace.

U druhého vzorku pro p-polarizaci odpovídá také velice přesně. Pro s-polarizaci je už ale znatelná odchylka, sic ne přílíš veliká. Navíc naměřená závislost je v některých pásmech vyšší a v některých nižší než teoretická (viz obrázek \ref{g:2}), takže není možné dojít k uspokojivému výsledku fitováním indexu lomu, protože funkce $R_s$ v \eqref{e:fresnely} je v $N$ všude rostoucí. Metoda nejmenších čtverců dává $N = \num{1.765}$.


Určení Brewsterova úhlu bylo překvapivě přesné. Dalo by se zpřesnit zvýšením citlivosti detektoru/zesílení signálu v okolí Brewsterova úhlu. Odečítali jsme totiž pouze jednu platnou číslici. Vztah \eqref{e:brewster} je velmi citlivý na chybu Brewsterova úhlu. Přesto se změřený index lomu velmi dobře shodoval s tím uvedeným na štítku.

%Závěr
\section*{Závěr}
Změřili jsme závislost intenzity odraženého světla na úhlu dopadu pro \emph{TE} i \emph{TM} polarizaci pro dva vzorky (viz grafy \ref{g:1} a \ref{g:2} a přiložené tabulka).

Naměřená data se dobře shodují s teorií.

Změřili jsme Brewsterův úhel obou vzorků
\begin{equation*}
\theta_{B1} = \SI{56.5(5)}{\degree} \qquad \qquad \qquad \theta_{B2} = \SI{61(1)}{\degree}
\end{equation*}
a z nich určili index lomu obou vzorků
\begin{equation*}
n_1 = \num{1.51(3)} \qquad \qquad \qquad n_2 = \num{1.80(7)} \,.
\end{equation*}

Na štítku byl uveden index lomu prvního vzorku \num{1.509} a druhého \num{1.8051}, takže naše hodnoty jsou v dobré shodě.


\printbibliography[title={Seznam použité literatury}]

\end{document}